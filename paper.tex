% This template has been tested with LLNCS DOCUMENT CLASS -- version 2.20 (24-JUN-2015)

%"runningheads" enables:
%  - page number on page 2 onwards
%  - title/authors on even/odd pages
%This is good for other readers to enable proper archiving among other papers and pointing to content.
%Even if the title page states the title, when printed and stored in a folder, when blindly opening the folder, one could hit not the title page, but an arbitrary page. Therefore, it is good to have title printed on the pages, too.
\documentclass[runningheads,a4paper]{llncs}

\usepackage{cmap}
\usepackage[T1]{fontenc}

%Even though `american`, `english` and `USenglish` are synonyms for babel package (according to https://tex.stackexchange.com/questions/12775/babel-english-american-usenglish), the llncs document class is prepared to avoid the overriding of certain names (such as "Abstract." -> "Abstract" or "Fig." -> "Figure") when using `english`, but not when using the other 2.
\usepackage[english]{babel}

%better font, similar to the default springer font
%cfr-lm is preferred over lmodern. Reasoning at http://tex.stackexchange.com/a/247543/9075
\usepackage[%
rm={oldstyle=false,proportional=true},%
sf={oldstyle=false,proportional=true},%
tt={oldstyle=false,proportional=true,variable=true},%
qt=false%
]{cfr-lm}
%
%if more space is needed, exchange cfr-lm by mathptmx
%\usepackage{mathptmx}

\usepackage{graphicx}


%% If you need packages for other papers,
%% START COPYING HERE

%extended enumerate, such as \begin{compactenum}
\usepackage{paralist}

%put figures inside a text
%\usepackage{picins}
%use
%\piccaptioninside
%\piccaption{...}
%\parpic[r]{\includegraphics ...}
%Text...

%Sorts the citations in the brackets
%It also allows \cite{refa, refb}. Otherwise, the document does not compile.
%  Error message: "White space in argument"
\usepackage{cite}

%for demonstration purposes only
\usepackage[math]{blindtext}

%for easy quotations: \enquote{text}
\usepackage{csquotes}

%enable margin kerning
\usepackage{microtype}

%tweak \url{...}
\usepackage{url}
\urlstyle{same}
%improve wrapping of URLs - hint by http://tex.stackexchange.com/a/10419/9075
\makeatletter
\g@addto@macro{\UrlBreaks}{\UrlOrds}
\makeatother
%nicer // - solution by http://tex.stackexchange.com/a/98470/9075
%DO NOT ACTIVATE -> prevents line breaks
%\makeatletter
%\def\Url@twoslashes{\mathchar`\/\@ifnextchar/{\kern-.2em}{}}
%\g@addto@macro\UrlSpecials{\do\/{\Url@twoslashes}}
%\makeatother

%diagonal lines in a table - http://tex.stackexchange.com/questions/17745/diagonal-lines-in-table-cell
%slashbox is not available in texlive (due to licensing) and also gives bad results. This, we use diagbox
%\usepackage{diagbox}

%required for pdfcomment later
\usepackage{xcolor}




%enable nice comments
\usepackage{pdfcomment}
%enable hyperref without colors and without bookmarks
\hypersetup{hidelinks,
   colorlinks=true,
   allcolors=black,
   pdfstartview=Fit,
   breaklinks=true}
%enables correct jumping to figures when referencing
\usepackage[all]{hypcap}

\newcommand{\commentontext}[2]{\colorbox{yellow!60}{#1}\pdfcomment[color={0.234 0.867 0.211},hoffset=-6pt,voffset=10pt,opacity=0.5]{#2}}
\newcommand{\commentatside}[1]{\pdfcomment[color={0.045 0.278 0.643},icon=Note]{#1}}

%compatibality with packages todo, easy-todo, todonotes
\newcommand{\todo}[1]{\commentatside{#1}}
%compatiblity with package fixmetodonotes
\newcommand{\TODO}[1]{\commentatside{#1}}

%enable \cref{...} and \Cref{...} instead of \ref: Type of reference included in the link
\usepackage[capitalise,nameinlink]{cleveref}
%Nice formats for \cref
\crefname{section}{Sect.}{Sect.}
\Crefname{section}{Section}{Sections}

\usepackage{xspace}
%\newcommand{\eg}{e.\,g.\xspace}
%\newcommand{\ie}{i.\,e.\xspace}
\newcommand{\eg}{e.\,g.,\ }
\newcommand{\ie}{i.\,e.,\ }

%introduce \powerset - hint by http://matheplanet.com/matheplanet/nuke/html/viewtopic.php?topic=136492&post_id=997377
\DeclareFontFamily{U}{MnSymbolC}{}
\DeclareSymbolFont{MnSyC}{U}{MnSymbolC}{m}{n}
\DeclareFontShape{U}{MnSymbolC}{m}{n}{
    <-6>  MnSymbolC5
   <6-7>  MnSymbolC6
   <7-8>  MnSymbolC7
   <8-9>  MnSymbolC8
   <9-10> MnSymbolC9
  <10-12> MnSymbolC10
  <12->   MnSymbolC12%
}{}
\DeclareMathSymbol{\powerset}{\mathord}{MnSyC}{180}

% correct bad hyphenation here
\hyphenation{op-tical net-works semi-conduc-tor}


%% END COPYING HERE


\begin{document}

\title{Paper Title}
%If Title is too long, use \titlerunning
%\titlerunning{Short Title}

%Single insitute
\author{Firstname Lastname \and Firstname Lastname}
%If there are too many authors, use \authorrunning
%\authorrunning{First Author et al.}
\institute{Institute}

%Multiple insitutes
%Currently disabled
%
\iffalse
%Multiple institutes are typeset as follows:
\author{Firstname Lastname\inst{1} \and Firstname Lastname\inst{2} }
%If there are too many authors, use \authorrunning
%\authorrunning{First Author et al.}

\institute{
Insitute 1\\
\email{...}\and
Insitute 2\\
\email{...}
}
\fi
			
\maketitle

\begin{abstract}
Abstract goes here
\end{abstract}

\begin{keywords}
keyword1, keyword2
\end{keywords}

%%%%%%%%%%%%%%%%%%%%%%%%%%%%%%%%%%%%%%%%%%%%%%%%%%%%%%%%%%%%%%%%%%%%%%%%%%%%%%%
\section{Introduction}\label{sec:intro}
%%%%%%%%%%%%%%%%%%%%%%%%%%%%%%%%%%%%%%%%%%%%%%%%%%%%%%%%%%%%%%%%%%%%%%%%%%%%%%%
\blindtext\todo{Refine me}

Winery~\cite{Winery} is graphical \commentontext{modeling}{modeling with one \enquote{l}, because of AE} tool.

\begin{figure}
Simple Figure
\caption{Simple Figure}
\label{fig:simple}
\end{figure}

\begin{table}
\caption{Simple Table}

\label{tab:simple}
Simple Table
\end{table}

cref Demonstration: Cref at beginning of sentence, cref in all other cases.

\Cref{fig:simple} shows a simple fact, although \cref{fig:simple} could also show something else.

\Cref{tab:simple} shows a simple fact, although \cref{tab:simple} could also show something else.

\Cref{sec:intro} shows a simple fact, although \cref{sec:intro} could also show something else.

Brackets work as designed:
<test>

The symbol for powerset is now correct: $\powerset$ and not a Weierstrass p ($\wp$).

\begin{inparaenum}
\item All these items...
\item ...appear in one line
\item This is enabled by the paralist package.
\end{inparaenum}

``something in quotes'' using plain tex or use \enquote{the enquote command}.

\section{Conclusion and Outlook}

\subsubsection*{Acknowledgments}
...

In the bibliography, use \texttt{\textbackslash textsuperscript} for ``st'', ``nd'', ...:
E.g., \enquote{The 2\textsuperscript{nd} conference on examples}.
When you use \href{https://www.jabref.org}{JabRef}, you can use the clean up command to achieve that.

%%%%%%%%%%%%%%%%%%%%%%%%%%%%%%%%%%%%%%%%%%%%%%%%%%%%%%%%%%%%%%%%%%%%%%%%%%%%%%%
\bibliographystyle{splncs03}
\bibliography{paper}

All links were last followed on October 5, 2014.
%%%%%%%%%%%%%%%%%%%%%%%%%%%%%%%%%%%%%%%%%%%%%%%%%%%%%%%%%%%%%%%%%%%%%%%%%%%%%%%

\end{document}
